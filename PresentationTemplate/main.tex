\documentclass{beamer}
\usepackage{settings}


%%%% Настройки титульного слайда %%%%
% внесите ваши данные
\titlegraphic{\includegraphics[height=1.5cm]{logo_isu.png}} % файл с лого
\institute{ФГБОУ ВО «Иркутский государственный университет»
        Факультет бизнес-коммуникаций и информатики\break
        Кафедра естественнонаучных дисциплин}
\title{\bfНазвание вашей работы} 
\author{
    {\bf Студент:} ФИО 14421-ДБ\\
    {\bf Консультант:} уч.степень, уч.звание или должность ФИО\\
    {\bf Научный руководитель:} уч.степень, уч.звание ФИО
}
\date{Иркутск 202*}



\begin{document}

%%%% Титульный слайд %%%%
\begin{frame}
    \titlepage 
\end{frame}


\numbered{ % заключаем слайды, на которых нужна нумерация в эту команду
%%%% Слайд с целью  вашей работы %%%%
\begin{frame}
    \frametitle{Цель}
    Здесь вы можете описать цель вашей работы.
\end{frame}


%%%% Слайд с задачами %%%%
\begin{frame}
    \frametitle{Задачи}
    \begin{itemize}
        \item Задача 1
        \item Задача 2
        \item Задача 3
        % Добавьте нужное количество задач
    \end{itemize}
\end{frame}


%%%% Шаблон слайда с таблицей %%%%
\begin{frame}
    \frametitle{Сравнительная таблица по выбору инструментов}
    \renewcommand{\arraystretch}{1}  % увеличение высоты ячеек в 1 раз
\newcolumntype{C}{ >{\centering\arraybackslash\footnotesize} m{2cm} }  % создаем новый тип столбца с вертикальным центрированием (m), чтобы не дублировать длинную запись (указать нужную ширину в сантиметрах)
\begin{table}[h]
    \caption{ Пример названия таблицы}  % закомментировать если не нужно 
    \begin{tabular}{|C|C|C|C|}  % описываем 3 столбца таблицы
    \hline  % горизонтальная черта
    &  Заголовок столбца 1 & Заголовок столбца 2 & Заголовок столбца 3 \\  
    \hline
    Заголовок строки 1 & Значение & Значение & Значение \\ \hline
    Заголовок строки 2 & Значение & Значение & Значение \\ \hline
    Заголовок строки 3 & Значение & Значение & Значение \\ \hline
    Заголовок строки 4 & Значение & Значение & Значение \\ \hline
    Заголовок строки 5 & Значение & Значение & Значение \\ \hline
    \end{tabular}
\end{table}
\end{frame}

%%%% Шаблон слайда со списком %%%%
\begin{frame}
    \frametitle{Использованные инструменты}
    \begin{itemize}
        \item инструмент 1;
        \item инструмент 2;
        \item инструмент 3.
    \end{itemize}
\end{frame}


%%%% Шаблон слайда с рисунком и подписью %%%%
\begin{frame}
    \frametitle{Рисунок с подписью}
    \vspace{15mm} % если надо сдвинуть рисунок пониже
    \begin{figure}
        \includegraphics[width=0.7\textwidth]{example-image}
        \caption{Подпись к рисунку}
    \end{figure}
\end{frame}


%%%% Шаблон слайда с двумя рисунками %%%%
\begin{frame}
    \frametitle{Два рисунка с подписями}
    \vspace{15mm} % если надо сдвинуть рисунки пониже
    \begin{figure}
        \centering
        \begin{minipage}[b]{0.45\textwidth}
            \includegraphics[width=\textwidth]{example-image-a}
            \caption{Подпись к рисунку A}
        \end{minipage}
        \hfill
        \begin{minipage}[b]{0.45\textwidth}
            \includegraphics[width=\textwidth]{example-image-b}
            \caption{Подпись к рисунку B}
        \end{minipage}
    \end{figure}
\end{frame}


%%%% Шаблон слайда c измененным размером заголовка %%%%
\begin{frame}
    \frametitle{\scriptsize Пример очень маленького заголовка}
    \vspace{5mm}
    
    \tiny самый маленький шрифт
    
    \scriptsize очень маленький шрифт
    
    \footnotesize шрифт для сносок
    
    \small маленький шрифт
    
    \normalsize базовый шрифт
    
    \large шрифт чуть больше базового
    
    \Large еще больше
    
    \LARGE очень большой шрифт
    
    \huge огромный шрифт
    
    \Huge самый большой шрифт
    
\end{frame}


%%%% Слайд с заключением вашей работы %%%%
\begin{frame}
    \frametitle{Заключение}
    Здесь вы можете сделать выводы по вашей работе.
\end{frame}


%%%% Слайд со списком использованных источников %%%%
\begin{frame}
\begin{onehalfspace}

    \frametitle{Список использованных источников}
\end{onehalfspace}
    \begin{enumerate}
        \item Источник 1.
        \item Источник 2.
        \item и т.д.
    \end{enumerate}
\end{frame}
}

%%%% Последний слайд %%%%
\unnumbered{ % убираем нумерацию на последнем слайде
\begin{frame}
    \frametitle{ }
    \centering
    \Huge
    Спасибо за внимание!
\end{frame}}

\end{document}
